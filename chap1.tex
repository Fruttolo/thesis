\chapter{Nozioni Introduttive}
Iniziamo introducendo alcune nozioni alla base di questa tesi: i \textbf{grafi}, strutture matematiche utilizzate nella rappresentazione dei documenti XML, e le relazioni di \textbf{Simulazione}, \textbf{Bisimulazione} e \textbf{Trace Equivalence}, relazioni binarie sui grafi che possono essere usate nella creazione di indici XML.

\section{I Grafi}
Introduciamo formalmente il concetto di grafo e le relative notazioni che utilizzeremo nel corso della tesi.
\\\\
I grafi sono strutture matematiche di grande importanza per la modellazione di svariati problemi in molteplici campi scientifici. Un grafo è composto da un insieme di entità astratte chiamate \textbf{nodi} e un insieme di coppie di nodi, detti \textbf{archi}, che sono in relazione tra loro.\\
Possiamo rappresentare graficamente un grafo come dei punti nello spazio, i nodi, connessi da delle linee, gli archi.

\begin{definition}[Grafo]
E' chiamato Grafo una coppia di insiemi $G = (N,E)$ dove $N$ è un insieme non vuoto di elementi detti nodi e $E \in N \times N $ è l'insieme degli elementi detti archi. Chiameremo $|N|$ \textbf{ordine} del grafo e $|E|$ \textbf{dimensione} del grafo.
\end{definition}

Dati due nodi $u, v \in N$ in un grafo $G = (N,E)$ diremo che esiste un cammino da $u$ a $v$ se esiste in $E$ un insieme di archi che ci permette di raggiungere il nodo, formalmente:

\begin{definition}[Cammino]
dati due nodi $u,v \in N$ esisterà un cammino $u \sim^k v$ di lunghezza $k$ se esiste una sequenza $<v_0,v_1,v_2,...,v_k>$ di vertici tale che la coppia $(v_{(i-1)},v_i) \in E$ per $i = 1,..,k$, con $u = v_0$ e $v = v_k$. 
\end{definition}


\begin{definition}[Ciclo]
E' definito ciclo, in un grafo, un cammino che va da un nodo v in se se stesso, ovvero $v \sim v$
\end{definition}

Possiamo dividere i grafi nel seguente modo in base alle proprietà che verificano :

\begin{itemize}
\item un grafo può essere \textbf{finito o infinto} a seconda della cardinalità di $N$.
\item un grafo si dice \textbf{etichettato} se ad ogni nodo e/o vertice è associata una etichetta di qualche tipo.
\item un grafo è \textbf{ciclico o aciclico} se contiene o meno cicli.
\item un grafo è detto \textbf{orientato} nel caso in cui gli archi abbiano un verso di percorrenza; un grafo semplice è non orientato ovvero $(u,v)\in E \implies (v,u) \in E$.
\end{itemize}

\begin{definition}[Grafo fortemente connesso]
Un grafo orientato è \textbf{fortemente connesso} se due vertici qualsiasi sono raggiungibili l'uno dall'altro, ovvero $\forall u, v \in \: N $, vale $ u \sim v $.
\end{definition}

\begin{definition}[Componente fortemente connessa]
Una \textbf{componente fortemente connessa} di un grafo orientato $G$ è un sottografo massimale di $G$ fortemente connesso in cui esiste un cammino orientato tra ogni coppia di nodi ad esso appartenenti.
\end{definition}
Le componenti fortemente connesse formano una partizione di $G$ poiché un nodo non può trovarsi contemporaneamente in due componenti fortemente connesse, di conseguenza un grafo orientato è fortemente connesso se e solo se ha una sola componente connessa.
\\\\
In seguito illustriamo alcuni esempi con i vari tipi di grafi introdotti in questo paragrafo:
\newpage
\begin{example} 
\begin{figure}[h]
\centering
\subfloat[]{
\begin{tikzpicture}[>=stealth',shorten >=1pt,scale=.4,node distance=2cm,
  thick,main node/.style={circle,fill=white,draw,font=\sffamily\Large\bfseries}]
  
   \node[main node] (1) {1};
  \node[main node] (2)[below right of=1] {2};
  \node[main node] (5) [right of=2]{5};
  \node[main node] (3) [below  of=2] {3};
  \node[main node] (4) [right of=3] {4};
  \node[main node] (6) [below right  of=5] {6};	  


  \path[every node/.style={font=\sffamily\small}] 
         (1) edge node [] {} (2)	  
  		 (2) edge node [] {} (3)
  		 	edge node [] {} (5)
  		 (3) edge node [] {} (4) 			
  		 (4) edge node [] {} (5)
  		 edge node [] {} (6)
  		 (5) edge node [] {} (6);
\end{tikzpicture}}
\qquad
\subfloat[]{
\begin{tikzpicture}[>=stealth',shorten >=1pt,scale=.4,node distance=2cm,
  thick,main node/.style={circle,fill=white,draw,font=\sffamily\Large\bfseries}]

 %\tikzset{ container/.style={draw , ellipse , dashed , inner sep =0.1em}}
  
  \node[main node] (1) {1};
  \node[main node] (5) [right of=1] {5};
  \node[main node] (4) [below left of=5]{4};
  \node[main node] (2) [below left of=1] {2};
  \node[main node] (6) [below right of=4] {6};
  \node[main node] (3) [below  of=2] {3};	  

%\node[container , fit=(2) (4)] (b1) {};
%\node at (b1.west) [left ,node distance=2 and 1] {B2: };
  \path[every node/.style={font=\sffamily\small}] 
         (1) edge node [] {} (2)	  
  		 (2) edge node [] {} (4)
  		 	edge node [] {} (3)
  		 (4) edge node [] {} (5)
  		 	edge node [] {} (6);
\end{tikzpicture}}
\caption{\label{fig:1}}
\end{figure}
\begin{center}
\textit{In figura \ref{fig:1} abbiamo due grafi non orientati etichettati sui nodi: il grafo (a) è ciclico, il grafo (b) è aciclico}
\end{center}
\end{example}

\begin{example}
\begin{figure}[h]
\centering
\subfloat[]
{
\begin{tikzpicture}[->,>=stealth',shorten >=1pt,auto,node distance=2cm,
  thick,main node/.style={circle,fill=white,draw,font=\sffamily\Large\bfseries}]
  
  \node[main node] (1) {1};
  \node[main node] (2) [below left of=1] {2};
  \node[main node] (3) [below right of=1] {3};
  \node[main node] (4) [below right of=2] {4};
  
  \path[every node/.style={font=\sffamily\small}] 
         (1) edge node [] {} (2)
         	 edge node [] {} (3)
         		  
  		 (2) edge node [] {} (3)
  		 	 edge node [] {} (4)
  		 (4) edge node [] {} (1);
  
 \end{tikzpicture}

}
\qquad
\subfloat[]
{
   \begin{tikzpicture}[->,>=stealth',shorten >=1pt,auto,node distance=2cm,
  thick,main node/.style={circle,fill=white,draw,font=\sffamily\Large\bfseries}]
  
  \node[main node] (1) {1};
  \node[main node] (2) [below left of=1] {2};
  \node[main node] (3) [below right of=1] {3};
  \node[main node] (4) [below right of=2] {4};
  
  \path[every node/.style={font=\sffamily\small}] 
         (1) edge node [] {} (2)
         	 edge node [] {} (3)
         	 edge node [] {} (4)	  
  		 (2) edge node [] {} (3)
  		 	 edge node [] {} (4);
 \end{tikzpicture}
}
\caption{\label{fig:2}}
\end{figure} 
\begin{center}
\textit{In figura \ref{fig:2} abbiamo due grafi orientati etichettati sui nodi: il grafo (a) è ciclico, il grafo (b) è aciclico}
\end{center}
\end{example}

\newpage

\begin{example}
\begin{figure}[h]
\centering
\begin{tikzpicture}[->,>=stealth',shorten >=1pt,auto,node distance=2cm,
  thick,main node/.style={circle,fill=white,draw,font=\sffamily\Large\bfseries}]
  
  \node[main node] (1) {1};
  \node[main node] (2) [below left of=1] {2};
  \node[main node] (3) [below right of=2] {3};
  \node[main node] (4) [right of=1] {4};
  \node[main node] (6) [below right of=4] {6};
  \node[main node] (5) [below left of=6] {5};
  
  \path[every node/.style={font=\sffamily\small}] 
         (1) edge node [] {} (3)
         	 edge node [] {} (4)	  
  		 (2) edge node [] {} (1)
  		 (3) edge node [] {} (2)	
  		 (4) edge node [] {} (5)
  		 (5) edge node [] {} (6)
  		 (6) edge node [] {} (4);
  
\end{tikzpicture}
\caption{\label{fig:3}}
\end{figure}
\begin{center}
\textit{In figura \ref{fig:3} abbiamo un grafo orientato etichettato sui nodi con due componenti fortemente connesse \{1,2,3\} e \{4,5,6\}.}
\end{center}
\end{example}

\subsection{Alberi}
Questa tesi si occuperà di un tipo particolare di grafo detto \textbf{albero}. Un albero è un grafo aciclico che verifica certe proprietà; questo nome deriva dal fatto che, fissato un nodo come radice, questo grafo può essere visto come un vero albero dalle cui radici partono varie ramificazioni (sotto-alberi).
\begin{definition}[Albero]
Si dice albero un grafo $G$ che verifica una della seguenti condizioni equivalenti:
\begin{itemize}
\item $G$ è aciclico e connesso.
\item aggiungere un arco a $G$ crea un ciclo.
\item se si rimuovere un arco $G$ non è più connesso.
\end{itemize}
\end{definition}

La struttura dati comunemente utilizzata in informatica è una tipologia di albero detto radicato \emph{radicato}, cioè un albero in cui un nodo viene scelto come radice.

\begin{definition}
Definiamo di seguito la radice e tutti i componenti dell'albero a partire da essa:\\
\begin{itemize}
\item La \textbf{radice} è il nodo di partenza. Tra la radice e qualsiasi nodo dell'albero c'è un percorso univoco (il senso del percorso è dato dall'orientamento dell'albero, generalmente è dalla radice al nodo).
\item Il \textbf{livello} è dato dalla lunghezza del percorso dal nodo alla radice, la radice ha livello 0. Il livello massimo raggiunto è detto \textbf{altezza} dell'albero. I nodi allo stesso livello sono detti \textbf{cugini}.
\item Dato un nodo i suoi \textbf{discendenti} sono tutti i nodi da esso raggiungibili allontanandosi dalla radice. Muovendoci verso la radice troveremo i suoi \textbf{antenati}.
\item Dato un nodo i suoi figli sono i \textbf{discendenti} al livello inferiore. Il \textbf{padre} di un nodo è l'antenato al livello superiore. I nodi figli dello stesso padre si dicono \textbf{fratelli}.
\item Una \textbf{foglia} è un nodo senza figli. Tale nodo è anche detto \textbf{nodo esterno}.
\end{itemize}
\end{definition}
Illustriamo di seguito la nozione di albero con un esempio.\\\\
%esempio albero
\begin{example}
\begin{figure}[H]
\centering
\begin{tikzpicture}[->,>=stealth',level/.style={sibling distance = 5cm/#1,
  level distance = 1.5cm}] 
\node [arn_n] {1}
    child{ node [arn_n] {2} 
            child{ node [arn_n] {4} 
            	child{ node [arn_n] {8}}							
            }
            child{ node [arn_n] {5}
							child{ node [arn_n] {9}}
            }                            
    }
    child{ node [arn_n] {3}
            child{ node [arn_n] {6} 
							child{ node [arn_n] {10}}
							child{ node [arn_n] {11}}
            }
            child{ node [arn_n] {7}
							child{ node [arn_n] {12}}
            }
		}
; 
\end{tikzpicture}
\caption{Esempio di Albero}\label{fig:albero}
\end{figure}
\begin{center}
\textit{Nella figura \ref{fig:albero} vediamo un esempio di albero. Il nodo 1 è la radice, i nodi 8,9,10,11,12 sono foglie al livello 3. Il nodo 3 è padre del nodo 6 e 7 e antenato di 6,7,10,11,12. 
}
\end{center}
\end{example}

\newpage
\section{Simulazione, Bisimulazione e Trace Equivalence}
In questa sezione introduciamo la \textbf{simulazione}, una relazione d'equivalenza sull'insieme dei nodi di un grafo e due relazione ad essa collegate: la \textbf{bisimulazione} e la \textbf{trace equivalence}.
\\\\
Le classi di equivalenza che costruiamo grazie a queste relazioni possono essere utilizzate per creare un grafo ridotto detto \textbf{grafo quoziente}, in cui intuitivamente i nodi simili vengono fusi in un unico nodo.\\
L'uso del grafo quoziente permette, a fronte di una perdita di informazione, di lavorare con strutture più piccole risparmiando tempo e risorse.\\
Tale perdita di  informazione è comunque controllata nel senso che i dati di interesse rispetto ad una classe specifica di linguaggi di interrogazione vengono preservati. Discuteremo di tali aspetti più avanti nella tesi.

\subsection{Simulazione}
Iniziamo definendo formalmente la relazione di simulazione. In seguito illustreremo la nozione con diversi esempi.

\begin{definition}[Relazione di Simulazione]
Sia $G = (N, E, \tau())$ un grafo etichettato dove con $\tau(v)$ indicheremo l'etichetta dei nel nodo v. Data la funzione $post(v) = \{u|(v,u) \in E\}$ che restituisce l'insieme dei successori del vertice $v$.
Possiamo definire una relazione binaria $\preceq\;\subseteq N^2$ detta \textbf{relazione di simulazione} se si verificano le seguenti condizioni:
\begin{enumerate}
\item $\tau(u) = \tau(v)$
\item $\forall u' \in post(u) \exists v' \in post(v) | u' \preceq_S v'$
\end{enumerate}
\begin{lemma}
Dato un grafo etichettato $G = (N,E,\tau())$ esiste un unica relazione di simulazione, denotata $\preceq_S$, massimo rispetto all'unione.
\end{lemma}
\end{definition}
\begin{definition}[Equivalenza di Simulazione]
Due vertici $u,v$ in grafo etichettato $G = (N,E)$ si dicono simili, denotato con $\equiv_S$, se $u\preceq_Sv$ e $v\preceq_Su$. La relazione $\equiv_S$ è una relazione di equivalenza sull'insieme $N$ dei nodi di $G$.
\end{definition}

\subsection{Bisimulazione}
Definiamo ora la relazione di bisimulazione
\begin{definition}[Relazione di Bisimulazione]
Dati due vertici $u$ e $v$ in grafo $G = (N,E,\tau())$, $u \simeq v$ se:
\begin{itemize}
\item $\tau(u) = \tau(v)$
\item $\forall\:u'\,\in\,post(u)\:\exists\:v'\,\in\,post(v)\,|\,u' \preceq_B v'$
\item $\forall\:v'\,\in\,post(v)\:\exists\:u'\,\in\,post(u)\,|\,v' \preceq_B u'$
\end{itemize}
\begin{lemma}
Dato un grafo etichettato $G = (N,E,\tau)$ esiste un unica relazione di bisimulazione, denotata $\preceq_B$, massimo rispetto all'unione.
\end{lemma}
\end{definition}
\begin{definition}[Equivalenza di Bisimulazione]
Due vertici $u,v$ in grafo etichettato $G = (N,E,\tau())$ si dicono bisimili, denotato con $\equiv_B$, se $u\preceq_Bv$ e $v\preceq_Bu$. La relazione $\equiv_B$ è una relazione di equivalenza sull'insieme $N$ dei nodi di $G$.
\end{definition}


\subsection{Trace Equivalence}
Illustriamo infine la trace equivalence, in figura possiamo vedere un confronto tra le due relazioni.
\begin{definition}
Dati due vertici $u$ e $v$ in grafo $G = (N,E)$, diciamo che $u$ domina $v$ se:
\begin{itemize}
\item definito $\bar{u}$ un cammino con radice in $u$
\item $\forall\,\bar{u}\quad\exists\,\bar{v}\;|\;\tau(\bar{u}) \,=\, \tau(\bar{v})$
\end{itemize}
La relazione di equivalenza $\equiv_T$ è detta Trace Equivalence e vale se $u$ domina $v$ e $v$ domina $u$.
\end{definition}
\begin{lemma}
La trace equivalence è implicata dalla simulazione ma non viceversa.
\end{lemma}
Questa relazione è molto importante perché due vertici equivalenti soddisfano le stesse formule di LTL e quindi il grafo quoziente $G/_{\approx^T}$ può essere usato per verificarle; tuttavia costituire questo grafo è molto difficile, il problema infatti è PSPACE-completo.
\\\\
%%%aESEMPIOOOOOOOOOO
\begin{example}
\begin{figure}[H]
\centering
\subfloat[Grafo G]{
\begin{tikzpicture}[->,>=stealth',shorten >=1pt,auto,node distance=2cm,
                    thick,main node/.style={circle,draw,font=\sffamily\Large\bfseries}]

  %\node[main node] (1) {/};
  \node[main node] (2) [%below of=1,
  label=right:5] {a};
  \node[main node] (3) [below left of=2,label=right:4] {b};
  \node[main node] (4) [below right of=2,label=right:6] {b};
  \node[main node] (5) [below of=4,label=right:7] {c};
  \node[main node] (6) [below of=5,label=right:8] {d};  
  \node[main node] (7) [below of=3,label=right:3] {c};
  \node[main node] (8) [below left of=3,label=right:1]{c};  
  \node[main node] (9) [below of=7,label=right:2]{d};
  
  \path[every node/.style={font=\sffamily\small}]
   % (1) edge node {} (2)
    (2) edge node {} (3)
        edge node {} (4)
    (3) edge node {} (7)
        edge node {} (8) 
    (4) edge node {} (5)
    (5) edge node {} (6)
    (7) edge node {} (9)
    (9) edge [loop left] node {} (9)
    (6) edge [loop left] node {} (6);
\end{tikzpicture}}
\qquad
\subfloat[Grafo $G/_{\equiv_S}$]{
\begin{tikzpicture}[scale=1,->,>=stealth',shorten >=1pt,auto,node distance=2cm,
                    thick,main node/.style={scale=1,circle,draw,font=\sffamily\Large\bfseries}]

  %\node[main node] (1) {/};
  \node[main node] (2) [%below of=1,
  label=right:5] {a};
  \node[main node] (3) [below of=2,label=right:4 6] {b};
  \node[main node] (4) [below left of=3,label=right:1] {c};
  \node[main node] (5) [below right of=3,label=right:3 7] {c};
  \node[main node] (6) [below of=5,label=right:2 8] {d};

  \path[every node/.style={font=\sffamily\small}]
    %(1) edge node {} (2)
    (2) edge node {} (3)
    (3) edge node {} (4)
        edge node {} (5) 
    (5) edge node {} (6)
    (6) edge [loop left] node {} (6);
\end{tikzpicture}}
\caption{Esempio di Simulazione.\label{fig:simulaizone}}
\end{figure}
\begin{center}
\textit{A sinistra abbiamo il grafo di partenza e a destra il grafo quoziente ricavato dalla relazione $\equiv_S$. Da notare come la cardinalità sia quasi dimezzata nel grafo quoziente.}
\end{center}
\end{example}
%ESAMPUIODSAAAAA
\begin{example}
\begin{figure}[H]
\centering
\subfloat[Grafo G]{
\begin{tikzpicture}[->,>=stealth',shorten >=1pt,auto,node distance=2cm,
                    thick,main node/.style={circle,draw,font=\sffamily\Large\bfseries}]

\node[main node] (1) [label=right:1] {r};
\node[main node] (2) [below left of=1,label=right:2] {a};
\node[main node] (3) [below right of=1,label=right:3] {a};
\node[main node] (4) [below of=2,label=right:4] {b};
\node[main node] (5) [below right of=2,label=right:5] {b};
\node[main node] (6) [below of=3,label=right:6] {b};
\node[main node] (7) [below of=4,label=right:7] {c};
\node[main node] (8) [below of=5,label=right:8] {c};
\node[main node] (9) [below of=6,label=right:9] {d};

\path[every node/.style={font=\sffamily\small}]
(1) edge node {} (2)
    edge node {} (3)
(2) edge node {} (4) edge node {} (5)
(3) edge node {} (6) edge node {} (5)
(4) edge node {} (7)
(5) edge node {} (8)
(6) edge node {} (9)    
(7) edge [loop left] node {} (7)
(8) edge [loop left] node {} (8)
(9) edge [loop left] node {} (9);
\end{tikzpicture}
}
\subfloat[Grafo $G/_{\approx^B}$]{
\begin{tikzpicture}[->,>=stealth',shorten >=1pt,auto,node distance=2cm,
                    thick,main node/.style={circle,draw,font=\sffamily\Large\bfseries}]

\node[main node] (1) [label=right:1] {r};
\node[main node] (2) [below left of=1,label=right:2] {a};
\node[main node] (3) [below right of=1,label=right:3] {a};
%\node[main node] (4) [below left of=2,label=right:4] {b};
\node[main node] (5) [below right of=2,label=right:4-5] {b};
\node[main node] (6) [below of=3,label=right:7-8] {b};
%\node[main node] (7) [below of=4,label=right:4] {c};
\node[main node] (8) [below of=5,label=right:6] {c};
\node[main node] (9) [below of=6,label=right:9] {d};

\path[every node/.style={font=\sffamily\small}]
(1) edge node {} (2)
    edge node {} (3)
(2) edge node {} (5)
(3) edge node {} (6) edge node {} (5)
%(4) edge node {} (7)
(5) edge node {} (8)
(6) edge node {} (9)    
%(7) edge [loop left] node {} (7)
(8) edge [loop left] node {} (8)
(9) edge [loop left] node {} (9);
\end{tikzpicture}
}
\subfloat[$G/_{\approx^T}$]{
\begin{tikzpicture}[->,>=stealth',shorten >=1pt,auto,node distance=2cm,
                    thick,main node/.style={circle,draw,font=\sffamily\Large\bfseries}]
\node[main node] (1) [label=right:4] {r};
\node[main node] (2) [below of=1,label=right:4] {a};
\node[main node] (3) [below of=2,label=right:4] {b};
\node[main node] (4) [below left of=3,label=right:4] {c};
\node[main node] (5) [below right of=3,label=right:4] {d};

\path[every node/.style={font=\sffamily\small}]
(1) edge node {} (2)
(2) edge node {} (3)
(3) edge node {} (4) edge node {} (5)
(4) edge [loop left] node {} (4)
(5) edge [loop left] node {} (5);
\end{tikzpicture}
}
\caption{Esempi di Trace Equivalence e Bisimulazione.\label{fig:bisim}}
\end{figure}
\begin{center}
\textit{A sinistra abbiamo un grafo G di partenza, al centro il grafo quoziente $G/_{\approx^B}$ e a destra il grafo quoziente $G/_{\approx^T}$. Notare come la Bisimulazione non ha grandi capacità di riduzione rispetto a trace equivalence e simulazione. Il grafo quoziente $G/_{\equiv_B}$ conserva un linguaggio più espressivo rispetto a quello ottenuto sfruttando la simulazione ma, essendo una relazione più fine, non fornisce una riduzione altrettanto importante in termini di spazio. Va notato come grazie all'algoritmo di Paige-Tarjan è possibile calcolare la bisimulazione in $O(mlogn)$.}
\end{center}
\end{example}